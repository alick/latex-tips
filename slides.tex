\documentclass[CJKchecksingle]{beamer}
% pass CJKchecksingle to xeCJK
% Author: alick<alick9188@gmail.com>
% 带中文支持的、使用 Beamer 的幻灯片模板
% 使用 xelatex 编译

% This file is modified from a solution template for:

% - Giving a talk on some subject.
% - The talk is between 15min and 45min long.
% - Style is ornate.

% Copyright 2004 by Till Tantau <tantau@users.sourceforge.net>.
%
% In principle, this file can be redistributed and/or modified under
% the terms of the GNU Public License, version 2.
%
% However, this file is supposed to be a template to be modified
% for your own needs. For this reason, if you use this file as a
% template and not specifically distribute it as part of a another
% package/program, I grant the extra permission to freely copy and
% modify this file as you see fit and even to delete this copyright
% notice.

\mode<presentation>
{
\usetheme[secheader]{Boadilla}
\setbeamercovered{transparent=5}
\usefonttheme[onlymath]{serif}
}

\usepackage{graphicx} % includegraphics support
%\graphicspath{{fig/}} % directories that hold graphics
\usepackage{listings} % for typesetting source code
\usepackage{amsmath} % for math
\usepackage{siunitx} % for si units
\usepackage{hologo}
\usepackage[style=numeric,sorting=none,firstinits=false]{biblatex}
\addbibresource{refs.bib}

\usepackage{fontspec}
%\usepackage[UTF8,nofonts]{ctex}
%
%% xeCJK conf setup
%\punctstyle{kaiming}
%\renewcommand\CJKfamilydefault{\CJKsfdefault} % for slides
%
%\setCJKmainfont[BoldFont={WenQuanYi Micro Hei},
%ItalicFont={AR PL UKai CN}]{AR PL UMing CN}
%\setCJKsansfont{WenQuanYi Micro Hei}
%\setCJKmonofont{WenQuanYi Micro Hei Mono}
%
%\setCJKfamilyfont{zhsong}{AR PL UMing CN}
%\setCJKfamilyfont{zhhei}{WenQuanYi Zen Hei}
%\setCJKfamilyfont{zhkai}{AR PL UKai CN}
%
%\newcommand*{\songti}{\CJKfamily{zhsong}} % 宋体
%\newcommand*{\heiti}{\CJKfamily{zhhei}}   % 黑体
%\newcommand*{\kaishu}{\CJKfamily{zhkai}}  % 楷书
%
%\renewcommand\tablename{表格}

\title[\LaTeX\ Notes]%
{Notes on Using \LaTeX}

\author[alick] % (optional, use only with lots of authors)
{Alick Zhao\\%
\url{alick9188@gmail.com}}

\institute[NiuLab] % (optional, but mostly needed)
{
NiuLab\\
Department of Electronic Engineering\\
Tsinghua University
}
% - Use the \inst command only if there are several affiliations.
% - Keep it simple, no one is interested in your street address.

\date % (optional)
{}

\subject{LaTeX, beamer, BibTeX}

% Delete this, if you do not want the table of contents to pop up at
% the beginning of each subsection:
\AtBeginSection[]
{
  \begin{frame}<beamer>{Outline}
    \tableofcontents[currentsection]
  \end{frame}
}


% If you wish to uncover everything in a step-wise fashion, uncomment
% the following command:

%\beamerdefaultoverlayspecification{<+->}

% listings setup
\lstset{%
language=[LaTeX]TeX,%
basicstyle=\ttfamily\color{blue!50!black},%
keywordstyle=\mdseries,%
commentstyle=\ttfamily,%
breaklines=true,%
morekeywords={alert,text},%
}
\lstdefinestyle{cli}{%
language=sh,
basicstyle=\ttfamily,%
}
\lstdefinestyle{wrong}{%
language=,
basicstyle=\ttfamily\color{gray},%
}
% hyperref setup
\hypersetup{
%pdfpagemode=FullScreen,
}

\lstMakeShortInline|
\newcommand{\env}[1]{\lstinline|#1|}
\newcommand{\pkg}[1]{\texttt{#1}}

\newcommand{\BibTeX}{\hologo{BibTeX}}
\newcommand{\beamer}{\textsc{beamer}}

\begin{document}

\begin{frame}
\titlepage
\end{frame}

\begin{frame}{Outline}
\tableofcontents
% You might wish to add the option [pausesections]
\end{frame}

% Since this a solution template for a generic talk, very little can
% be said about how it should be structured. However, the talk length
% of between 15min and 45min and the theme suggest that you stick to
% the following rules:

% - Exactly two or three sections (other than the summary).
% - At *most* three subsections per section.
% - Talk about 30s to 2min per frame. So there should be between about
%   15 and 30 frames, all told.

%\begin{frame}{Picture}
%  \begin{figure}[h]
%    \centering
%    \includegraphics[scale=0.4]{foo.png}
%    \caption{}
%  \end{figure}
%\end{frame}

%\begin{frame}{Why \TeX\ (and why not)?}
%  \begin{columns}[t]
%    \begin{column}{.45\textwidth}
%      \begin{block}{Pros}
%        \begin{itemize}
%          \item allow focusing on the content without bothering the
%            layout(including TOC, references, etc)
%          \item excel in typesetting math formular
%          \item enhanced in many aspects
%        \end{itemize}
%      \end{block}
%    \end{column}
%
%    \begin{column}{.45\textwidth}
%      \begin{block}{Cons}
%        \begin{itemize}
%          \item not WYSIWYG
%          \item not easy to be \TeX{}pert
%            \bigskip
%          \item only .doc permitted
%        \end{itemize}
%      \end{block}
%    \end{column}
%
%  \end{columns}
%\end{frame}


\section{Basics}

\subsection{Paragraphs}

\begin{frame}[fragile]\frametitle{Sectioning the contents.}
\begin{table}[h]
  \centering
  \caption{Sectioning commands~\cite{leoliu}.}
  \begin{tabular}{rlp{.6\textwidth}}
    \hline
    Level & Command & Comment \\
    \hline
    -1 & |\part| & Optional top level. \\
    0  & |\chapter| & Top level for documentclass \pkg{report}, \pkg{book}. \\
    1  & |\section| & Top level for documentclass \pkg{article}. \\
    2  & |\subsection| & \\
    3  & |\subsubsection| & Neither numbered nor shown in \pkg{report}, \pkg{book} by default.\\
    4  & |\paragraph| & Neither numbered nor shown by default.\\
    5  & |\subparagraph| & Neither numbered nor shown by default.\\
    \hline
  \end{tabular}
\end{table}

  \begin{itemize}
    \item |\appendix|
    \item Documentclass \pkg{book}: |\frontmatter|, |\mainmatter|,
      |\backmatter|
  \end{itemize}
\end{frame}

\begin{frame}[fragile]\frametitle{Do not abuse \lstinline|\\\\|.}

\begin{itemize}
\item
  Not for new paragraph. Use blank lines instead.
\item
  Not commonly used for newline in paragraph. Let \TeX\ hyphen.
\pause
\item
  Used inside \lstinline{\title} to tune word breaking.
\item
  Used inside \lstinline{\tabular}, multi-line formulas, etc.
\end{itemize}

\end{frame}

\subsection{Figures}

\begin{frame}\frametitle{Insert bitmap (PNG, JPG) figures.}

\begin{itemize}
\item
  The whole figure file will be swallowed in the PDF.
\item
  DPI mismatch leads to blurring.
\pause
\item
  Solutions~\cite{hoeppner,blurry,blurry2}:
  \begin{itemize}
    \item
      Line drawings: \alert{vector graphics}, high dpi PNG
    \item
      Photos: (high resolution) JPEG should be OK
    \item
      Screenshots: zoom out first
  \end{itemize}
\end{itemize}

\end{frame}

\begin{frame}[fragile]\frametitle{Specify figure/table/minipage size.}

\begin{itemize}
\item
  Prefer relative length to absolute length.
\item
  Prefer \lstinline{.45\textwidth} to \lstinline{2in}, \lstinline{3cm}.
\item
  \lstinline{\textheight}, \lstinline{\paperheight}, \lstinline{\paperwidth}
\end{itemize}

\end{frame}

\begin{frame}[fragile]\frametitle{Deal with subfigures.}

\begin{itemize}
\item
  Prefer \pkg{subcaption} package.
\item
  Use \pkg{subfig} with \pkg{IEEEtran} articles.
\item
  Do not use deprecated \pkg{subfigure} package.
\end{itemize}

\end{frame}

\subsection{Math \& Science}

\begin{frame}[fragile]\frametitle{Compose math formulas.}

\begin{itemize}
\item
  Variables: use common notations, introduce them
\item
  Vectors/Matrices in bold: \pkg{bm}, \lstinline{\mathbf}
\item
  Non-variables should not be \textit{italic}:
  \begin{itemize}
    \item |R_{\max}| $R_{\max}$
    \item |\text{SINR}| $\text{SINR}$
    \item |\mathrm{e}^{\mathrm{i}\pi}+1=0|
      $\mathrm{e}^{\mathrm{i}\pi}+1=0$
  \end{itemize}
\item
  Proper brackets size: |\left( \sum \ldots \right)|
\end{itemize}

\end{frame}

\begin{frame}[fragile]\frametitle{Handle long math formulas.}

\begin{itemize}
\item
  Define intermediate variables/functions to rephrase.
\item
  Divide it into multiple lines.
\item
  Environments: \env{multline}, \env{split}, \env{array}
\item
  Match bracket size: |\vphantom|
\end{itemize}

\end{frame}

\begin{frame}[fragile]\frametitle{Write numbers with units.}

\begin{itemize}
\item
  \pkg{siunitx} package:
  \begin{itemize}
    \item |\SI{9.8}{m/s^2}| \SI{9.8}{m/s^2}
    \item |\SI{6.02e-23}{mol^{-1}}| \SI{6.02e-23}{mol^{-1}}
  \end{itemize}
\item
  If \pkg{siunitx} is not possible:
  \begin{itemize}
    \item Write \lstinline{5~km}, not \lstinline{5km}.
  \end{itemize}
\end{itemize}

\end{frame}

\section{\BibTeX\ References}
\subsection{Database}
\begin{frame}[fragile]
  \frametitle{Do not blindly copy bib export from websites! I}

  \begin{itemize}
    \item Common errors~\cite{biberrors}:
      \begin{itemize}
        \item<1->
          Author names: \lstinline[style=wrong]|Owens, J.D.|
          |John D. Owens|
        \item<2->
          Title case of title, booktitle
\begin{lstlisting}[style=wrong]
booktitle = {Proceedings of the second workshop on Software radio implementation forum},
\end{lstlisting}
\begin{lstlisting}
booktitle = {Proceedings of the Second Workshop on Software Radio Implementation Forum},
\end{lstlisting}%
        \item<3->
          Booktitle word order
\begin{lstlisting}[style=wrong]
booktitle={Intelligent Vehicles Symposium (IV), 2011 IEEE},
\end{lstlisting}
\begin{lstlisting}
booktitle = {Proceedings of the 2011 IEEE Intelligent Vehicles Symposium},
\end{lstlisting}
\end{itemize}
\end{itemize}
\end{frame}
\begin{frame}[fragile]
  \frametitle{Do not blindly copy bib export from websites! II}
  \begin{itemize}
    \item Common errors continued:
      \begin{itemize}
        \item
          Month, pages
\begin{lstlisting}[style=wrong]
month={june},
pages={195 -200},
\end{lstlisting}
\begin{lstlisting}
month=jun,
pages={195--200},
\end{lstlisting}
\pause
        \item
          DOI, URL
\begin{lstlisting}[style=wrong]
url = {http://doi.acm.org/10.1145/1964179.1964185},
doi = {http://doi.acm.org/10.1145/1964179.1964185},
\end{lstlisting}
\begin{lstlisting}
doi = {10.1145/1964179.1964185},
\end{lstlisting}
      \end{itemize}
    \pause
    \item
      Auto fix: \href{https://github.com/alick9188/fixbib}{FixBib
      for Greasemonkey}
  \end{itemize}

\end{frame}

\begin{frame}[fragile]\frametitle{Deal with bachelor thesis, patent, and others}

\begin{itemize}
  \item Bachelor's thesis: |masterthesis| entry with |type = {Bachelor's Thesis}|
  \item White paper: |techreport| entry with |type = {White Paper}|
  \item Patent, standard: \lstinline[style=cli]|IEEEtran.bst|
    extension, \lstinline{misc} entry as fallback
  \item Foreign references: \lstinline|note = {In Chinese.}|
\end{itemize}

\end{frame}


\subsection{Citation}
\begin{frame}[fragile]\frametitle{Citation}

\begin{itemize}
\item
  Use tie (tilde, nbsp): \lstinline|Text goes~\cite{texbook}.|
\item
  Cite multiple references: |\cite{ref1,ref2,ref3}|
\pause
\item
  Can citations in brackets used as text words?
\end{itemize}

\end{frame}

\subsection{Compilation}

\begin{frame}[fragile]\frametitle{How many times to run?}

  \begin{figure}[htbp]
    \centering
    \includegraphics[width=\textwidth]{bibcompilation}
    \caption{\BibTeX\ compilation process.}
  \end{figure}

\begin{itemize}
    \pause
\item Automatic compilation:
  \begin{itemize}
    \item \lstinline[style=cli]{latexmk}, \lstinline[style=cli]|texify|
    \item Makefile
  \end{itemize}
\end{itemize}

\end{frame}

\section{\beamer\ Slides}
\subsection{Basics}
\begin{frame}[fragile]\frametitle{Fonts}

\begin{itemize}
\item
  Font family~\cite{beamer}: serif or sans-serif?
\begin{itemize}
  \item
    Sans is preferred today.
  \item
    Serif may be used in conservative scenario.
  \item
    Math texts might be better in serif.
    |\usefonttheme[onlymath]{serif}|
\end{itemize}
\pause
\item
  Prefer colored to italic texts.
\begin{itemize}
  \item |\alert{dangerours}| \alert{dangerours}
  \item |\emph{dangerours}| \emph{dangerours}
\end{itemize}
\end{itemize}

\end{frame}

\begin{frame}[fragile]\frametitle{Fragile frames}
  \begin{itemize}
    \item |\verb+Lorem Lipsum+|, \env{verbatim}
    \item |\begin{frame}[fragile] ... \end{frame}|
  \end{itemize}
\end{frame}

\begin{frame}[fragile]\frametitle{Chinese}

\begin{lstlisting}
\documentclass[CJKchecksingle]{beamer}
\usepackage[UTF8,nofonts]{ctex}
\renewcommand\CJKfamilydefault{\CJKsfdefault}

\setCJKmainfont[BoldFont={WenQuanYi Micro Hei},
ItalicFont={AR PL UKai CN}]{AR PL UMing CN}
\setCJKsansfont{WenQuanYi Micro Hei}
\setCJKmonofont{WenQuanYi Micro Hei Mono}

\setCJKfamilyfont{zhsong}{AR PL UMing CN}
\setCJKfamilyfont{zhhei}{WenQuanYi Zen Hei}
\setCJKfamilyfont{zhkai}{AR PL UKai CN}
\newcommand*{\songti}{\CJKfamily{zhsong}}
\newcommand*{\heiti}{\CJKfamily{zhhei}}
\newcommand*{\kaishu}{\CJKfamily{zhkai}}
\end{lstlisting}

\end{frame}

\subsection{Long Slides}
\begin{frame}[fragile]\frametitle{Show (current) page numbers.}
  \begin{itemize}
    \item |\usetheme{Boadilla} % or Madrid, AnnArbor, ...|
    \item |\useoutertheme{infolines}|
    \item Simple customization:
\begin{lstlisting}
\setbeamertemplate{footline}
{
  \leavevmode%
  \hbox{%
  \begin{beamercolorbox}[wd=\paperwidth,ht=2.25ex,dp=1ex,right]{page number in head/foot}%
    \insertframenumber{} / \inserttotalframenumber\hspace*{2ex}
  \end{beamercolorbox}}%
  \vskip0pt%
}
\end{lstlisting}
  \end{itemize}
\end{frame}

\subsection{Customization}

\begin{frame}[fragile,allowframebreaks]\frametitle{Footnote citations?}
% Provide \notefullcite for footnote citations.
% cf. http://www.texdev.net/2010/03/08/biblatex-numbered-citations-as-footnotes/

% NOTE: for vanilla footnote coexistence, refer to
% http://tex.stackexchange.com/questions/20754/tuning-numbered-citations-as-footnote
% http://tex.stackexchange.com/questions/20787/biblatex-cite-with-footnote-only-once-with-use-of-brackets
\begin{lstlisting}
\usepackage[style=ieee]{biblatex}
\ExecuteBibliographyOptions{citetracker=true,sorting=none}
\AtEveryCitekey{%
  \clearlist{publisher}%
  \clearfield{note}}

\DeclareFieldFormat{labelnumberwidth}{\mkbibbrackets{#1}}

\makeatletter
\renewcommand\@makefntext[1]{%
  \normalfont[\@thefnmark]\enspace #1}
\makeatother

\DeclareCiteCommand{\notefullcite}[\mkbibbrackets]
  {\usebibmacro{cite:init}%
   \usebibmacro{prenote}}
  {\usebibmacro{citeindex}%
   \usebibmacro{notefullcite}%
   \usebibmacro{cite:comp}}
  {}
  {\usebibmacro{cite:dump}%
   \usebibmacro{postnote}}
\newbibmacro*{notefullcite}{%
  \ifciteseen
    {}
    {\footnotetext[\thefield{labelnumber}]{%
       \usedriver{}{\thefield{entrytype}}.}}}


% Decrease the footnote size.
\let\oldfootnotesize\footnotesize
\renewcommand*{\footnotesize}{\oldfootnotesize\tiny}

% Disable the navigation symbol bar.
\beamertemplatenavigationsymbolsempty

% Add bib database.
\addbibresource{refs.bib}
\end{lstlisting}
\end{frame}

\begin{frame}[fragile]\frametitle{Create new templates}

\begin{itemize}
\item
  Background pictures: |\usebackgroundtemplate|
\item
  Logo: |\logo|
\item
  Directory structure
\begin{lstlisting}[style=cli]
$ ls themes/
color  font  inner  outer  theme
\end{lstlisting}
\pause
\item
  THU slides template?
\end{itemize}

\end{frame}

\section{Summary}

\begin{frame}\frametitle{Summary}

\begin{itemize}
\item
  Details matter for aesthetic and high quality documents.
\pause
\item
  \LaTeX\ is evolving. The old will be replaced by the new.
\pause
\item
  Some work is tedious. Try automate it!
\pause
\item
  Happy \TeX{}ing!
\end{itemize}
\nocite{l2tabu,elements}

\end{frame}

\appendix
\section<presentation>*{\appendixname}
\subsection<presentation>*{References}

\begin{frame}[allowframebreaks]
  \frametitle<presentation>{More readings.}

  \printbibliography

\end{frame}

\subsection<presentation>*{Join}
\begin{frame}
  \frametitle{Join}
  \begin{itemize}
    \item ThuThesis: \url{https://github.com/xueruini/thuthesis}
    \item ThuSlides
    \item TUNA: \url{http://tuna.tsinghua.edu.cn/}
  \end{itemize}
\end{frame}

\end{document}
%%% vim: set sw=2 isk+=\: et tw=70 formatoptions+=mB:

\documentclass[CJKchecksingle]{beamer}
% pass CJKchecksingle to xeCJK
% Author: alick<alick9188@gmail.com>
% This file is modified from a solution template for:

% - Giving a talk on some subject.
% - The talk is between 15min and 45min long.
% - Style is ornate.

% Copyright 2004 by Till Tantau <tantau@users.sourceforge.net>.
%
% In principle, this file can be redistributed and/or modified under
% the terms of the GNU Public License, version 2.
%
% However, this file is supposed to be a template to be modified
% for your own needs. For this reason, if you use this file as a
% template and not specifically distribute it as part of a another
% package/program, I grant the extra permission to freely copy and
% modify this file as you see fit and even to delete this copyright
% notice.

\mode<presentation>
{
\usetheme{Madrid}
\usefonttheme[onlymath]{serif}
\setbeamercovered{transparent=5}
\definecolor{fedorablue}{RGB}{60,110,180}
\definecolor{fedoradarkblue}{RGB}{41,65,114}
\definecolor{fedoradarkgrey}{RGB}{76,76,76}
\setbeamercolor*{palette primary}{fg=white,bg=fedorablue}
\setbeamercolor*{palette secondary}{fg=white,bg=fedoradarkblue}
\setbeamercolor*{palette tertiary}{fg=white,bg=fedorablue}
\setbeamercolor*{palette quaternary}{fg=white,bg=black}
}

\usepackage{graphicx} % includegraphics support
%\graphicspath{{fig/}} % directories that hold graphics
\usepackage{listings} % for typesetting source code
\usepackage{amsmath} % for math
\usepackage{siunitx} % for si units
\usepackage{hologo}
\usepackage[style=numeric,sorting=none,firstinits=false]{biblatex}
\addbibresource{refs.bib}

\usepackage{hyperxmp}
\hypersetup{
pdfauthor={Alick Zhao},
pdfcopyright={Copyright (C) 2014 by Alick Zhao.
Licensed under CC-BY-SA 4.0. Some rights reserved.},
pdflicenseurl={http://creativecommons.org/licenses/by-sa/4.0/},
}

\usepackage{fontspec}
\usepackage[UTF8,nofonts]{ctex}

% xeCJK conf setup
\punctstyle{kaiming}
\renewcommand\familydefault{\sfdefault} % for slides

\setCJKmainfont[BoldFont={WenQuanYi Micro Hei},
ItalicFont={AR PL UKai CN}]{AR PL UMing CN}
\setCJKsansfont{WenQuanYi Micro Hei}
\setCJKmonofont{WenQuanYi Micro Hei Mono}

\setCJKfamilyfont{zhsong}{AR PL UMing CN}
\setCJKfamilyfont{zhhei}{WenQuanYi Zen Hei}
\setCJKfamilyfont{zhkai}{AR PL UKai CN}

\newcommand*{\songti}{\CJKfamily{zhsong}} % 宋体
\newcommand*{\heiti}{\CJKfamily{zhhei}}   % 黑体
\newcommand*{\kaishu}{\CJKfamily{zhkai}}  % 楷书

%\renewcommand\tablename{表格}

%Workshop (training)

%LaTeX can create high quality outputs, but you need to be careful to avoid common errors. I want to do an interactive workshop/training to walk through common mistakes in LaTeX usage, and more importantly talk about the correct way. I will also talk about how to troubleshoot the "mysterious" LaTeX error log.   I will use a slides, and I want the audience to have computers with LaTeX installed so it is better to be on the next day of the LaTeX talk ;)

%I am a Fedora contributor, a long time LaTeX lover, and a college student.

%This talk serves to improve the LaTeX skills of attendees by pointing out common errors and how to debug. It can help make LaTeX yet another productive tool for them."

\title[\LaTeX\ Tips]%
{\LaTeX\ Tips}
\subtitle{Common Errors, How to Debug, and How to Correct}

\author[alick] % (optional, use only with lots of authors)
{Zhao Tao\\ \texttt{alick@fedoraproject.org}}

\institute[Fedora] % (optional, but mostly needed)
{
\includegraphics[width=.05\textwidth]{wu.jpg}\\
\inst{1}Fedora Project
\and
\inst{2}Tsinghua University TUNA Asscociation (TUNA)
\and
\inst{3}Tsinghua University
}
% - Use the \inst command only if there are several affiliations.
% - Keep it simple, no one is interested in your street address.

\date[FUDCon APAC 2014] % (optional)
{May 25, 2014}

\subject{LaTeX, beamer, BibTeX}

% Delete this, if you do not want the table of contents to pop up at
% the beginning of each subsection:
\AtBeginSection[]
{
  \begin{frame}<beamer>{Outline}
    \tableofcontents[currentsection]
  \end{frame}
}


% If you wish to uncover everything in a step-wise fashion, uncomment
% the following command:

%\beamerdefaultoverlayspecification{<+->}

% listings setup
\lstset{%
language=[LaTeX]TeX,%
basicstyle=\ttfamily\color{blue!50!black},%
keywordstyle=\mdseries,%
commentstyle=\ttfamily,%
breaklines=true,%
morekeywords={alert,text},%
}
\lstdefinestyle{cli}{%
language=sh,
basicstyle=\ttfamily,%
}
\lstdefinestyle{wrong}{%
language=,
basicstyle=\ttfamily\color{gray},%
}
% hyperref setup
\hypersetup{
%pdfpagemode=FullScreen,
}

\lstMakeShortInline|
\newcommand{\env}[1]{\lstinline|#1|}
\newcommand{\pkg}[1]{\texttt{#1}}

\newcommand{\pdfTeX}{\hologo{pdfTeX}}
\newcommand{\BibTeX}{\hologo{BibTeX}}
\newcommand{\beamer}{\textsc{beamer}}

\logo{\includegraphics[height=.1\textheight]{Logo_fedoralogo.png}}

% Disable the navigation symbol bar.
\beamertemplatenavigationsymbolsempty

\begin{document}

\begin{frame}
\titlepage
\end{frame}

\begin{frame}{Outline}
\tableofcontents
% You might wish to add the option [pausesections]
\end{frame}

% Since this a solution template for a generic talk, very little can
% be said about how it should be structured. However, the talk length
% of between 15min and 45min and the theme suggest that you stick to
% the following rules:

% - Exactly two or three sections (other than the summary).
% - At *most* three subsections per section.
% - Talk about 30s to 2min per frame. So there should be between about
%   15 and 30 frames, all told.

%\begin{frame}{Picture}
%  \begin{figure}[h]
%    \centering
%    \includegraphics[scale=0.4]{foo.png}
%    \caption{}
%  \end{figure}
%\end{frame}

%\begin{frame}{Why \TeX\ (and why not)?}
%  \begin{columns}[t]
%    \begin{column}{.45\textwidth}
%      \begin{block}{Pros}
%        \begin{itemize}
%          \item allow focusing on the content without bothering the
%            layout(including TOC, references, etc)
%          \item excel in typesetting math formular
%          \item enhanced in many aspects
%        \end{itemize}
%      \end{block}
%    \end{column}
%
%    \begin{column}{.45\textwidth}
%      \begin{block}{Cons}
%        \begin{itemize}
%          \item not WYSIWYG
%          \item not easy to be \TeX{}pert
%            \bigskip
%          \item only .doc permitted
%        \end{itemize}
%      \end{block}
%    \end{column}
%
%  \end{columns}
%\end{frame}

\section{Introduction}

% LaTeX
\begin{frame}[fragile]{\TeX\ and \LaTeX}
  \begin{itemize}
    \item \LaTeX{}: A Document Preparation System
      \begin{itemize}
        \item Created by Leslie Lamport in early 1980s
        \item Based on \TeX\ by Donald E.~Knuth
        \item Not word processor, not WYSIWYG, compilation
        \item Free and open source (LPPL)
      \end{itemize}
    \item Concepts and Terms
      \begin{itemize}
        \item Document class: generic template (article, book, report)
        \item Packages: add-ons (\pkg{graphicx}, \pkg{listings})
        \item Environment: |\begin{abstract}...\end{abstract}|
        \item Command: |\section{Introduction}|
        \item Engine: \hologo{XeTeX}, |xelatex|
        \item Format: LaTeX, Plain TeX, ConTeXt
      \end{itemize}
  \end{itemize}
\end{frame}

% content style separation
\begin{frame}[fragile]{Separation of Presentation and Content}
\begin{lstlisting}
\documentclass{article}
% Preamble: style configurations
\usepackage{hyperref}
\hypersetup{breaklinks=false}
\begin{document}
% Content:
My masterpiece.
\end{document}
\end{lstlisting}
\end{frame}

\begin{frame}[fragile]\frametitle{Chinese 中文支持}

\begin{lstlisting}
\documentclass[CJKchecksingle]{beamer}
\usepackage[UTF8,nofonts]{ctex}
\renewcommand\familydefault{\sfdefault}

\setCJKmainfont[BoldFont={WenQuanYi Micro Hei},
ItalicFont={AR PL UKai CN}]{AR PL UMing CN}
\setCJKsansfont{WenQuanYi Micro Hei}
\setCJKmonofont{WenQuanYi Micro Hei Mono}

\setCJKfamilyfont{zhsong}{AR PL UMing CN}
\setCJKfamilyfont{zhhei}{WenQuanYi Zen Hei}
\setCJKfamilyfont{zhkai}{AR PL UKai CN}
\newcommand*{\songti}{\CJKfamily{zhsong}}
\newcommand*{\heiti}{\CJKfamily{zhhei}}
\newcommand*{\kaishu}{\CJKfamily{zhkai}}
\end{lstlisting}

\end{frame}
\section{Compilation Errors}

% log
\begin{frame}[fragile]{Log is your friend.}
  \begin{itemize}
    \item Log
      \begin{itemize}
        \item On screen \& in |.log| file
        \item engine/document class/package version
        \item Error messages
\begin{lstlisting}
! Undefined control sequence.
l.3 \secton
           {Start}
?
\end{lstlisting}
        \item Interactive: ?, H, X, I
      \end{itemize}
  \end{itemize}
  \nocite{liu2013latex}
\end{frame}

\begin{frame}[fragile]{\TeX\ Errors}
  \begin{itemize}
    \item |! Undefined control sequence.|
      \begin{itemize}
        \item Typo
        \item Forget to |\usepackage{...}|
        \item Misremember the command
      \end{itemize}
    \item |! Double subscript.|
      \begin{itemize}
        \item |a_b_c| $\rightarrow$ |a_{b_c}| or |a_{bc}|
      \end{itemize}
    \item |! Extra alignment tab has been changed to \cr.|
      \begin{itemize}
        \item Write too many |&|s
        \item Forget add trailing |\\\\|
      \end{itemize}
    \item |!TeX capacity exceeded, sorry [...].|
      \begin{itemize}
        \item Recursive macro definition
        \item Tune memory setting
      \end{itemize}
  \end{itemize}
\end{frame}

\begin{frame}[fragile]{\LaTeX\ Errors}
  \begin{itemize}
    \item |Command ... already defined|
      \begin{itemize}
        \item |\newcommand| $\rightarrow$ |\renewcommand|
      \end{itemize}
    \item |Unknown graphics extension ...| or
      |Cannot determine size of graphic in ... (no BoundingBox).|
      \begin{itemize}
        \item Use EPS figure with \pdfTeX\ without automatic epstopdf support
        \item Bad EPS figure generated by some tools
      \end{itemize}
    \item |There's no line here to end.|
      \begin{itemize}
        \item Extra |\\\\| or |\newline| $\rightarrow$ |\vspace|
      \end{itemize}
  \end{itemize}
\end{frame}

\begin{frame}[fragile]{\TeX\ Warnings}
  \begin{itemize}
    \item Overfull/Underfull |\hbox|/|\vbox|
      \begin{itemize}
        \item Unreasonable use of |\newline|
        \item Adjust content
        \item Use floating env for figures/tables
        \item Insert |\newpage| to tune manually
      \end{itemize}
  \end{itemize}
\end{frame}

\begin{frame}[fragile]{\LaTeX\ Warnings}
  \begin{itemize}
    \item Citation/Reference |`...' on page ... undefined.|
      \begin{itemize}
        \item Normal if compilation is not final
        \item Error otherwise
      \end{itemize}
    \item Label |`...' multiply defined.|
      \begin{itemize}
        \item Fix label definition.
      \end{itemize}
  \end{itemize}
\end{frame}

\begin{frame}[fragile]{Debugging tools}
  \begin{itemize}
    \item Complex document: bisect
      \begin{itemize}
        \item |\includeonly|
        \item Post-amble
        \item Block comment or \env{comment} from \pkg{verbatim}
      \end{itemize}
    \item Show definition
      \begin{itemize}
        \item |\show|
        \item \pkg{show2e}
      \end{itemize}
  \end{itemize}
\end{frame}

\section{Bad Outputs}

\begin{frame}[fragile]\frametitle{Do not abuse \lstinline|\\\\|.}

\begin{itemize}
\item
  Not for new paragraph. Use blank lines instead.
\item
  Not commonly used for newline in paragraph. Let \TeX\ hyphen.
  (|\-| to guide hyphenation)
\pause
\item
  Used inside \lstinline{\title} to tune word breaking.
\item
  Used inside \env{tabular}, multi-line formulas, etc.
\end{itemize}

\end{frame}

\begin{frame}[fragile]\frametitle{Be careful with punctuations.}
  \begin{itemize}
    \item Quotation: |``| and |''|
    \item Period:
      \begin{itemize}
        \item |Dr.~Zhou|
        \item |Li et al.\ proposed ...|
        \item |... Mac OS X\@. New ...|
      \end{itemize}
    \item Dash: |-|, |--|, and |---|
    \item Serial comma: |... foo, bar, and baz ...|
  \end{itemize}
\end{frame}


% figure positioning

\begin{frame}\frametitle{Insert bitmap (PNG, JPG) figures.}

\begin{itemize}
\item
  The whole figure file will be swallowed in the PDF.
\item
  DPI mismatch leads to blurring.
\pause
\item
  Solutions~\cite{hoeppner,blurry,blurry2}:
  \begin{itemize}
    \item
      Line drawings: \alert{vector graphics}, high dpi PNG
    \item
      Photos: (high resolution) JPEG should be OK
    \item
      Screenshots: zoom in first
  \end{itemize}
\end{itemize}

\end{frame}

\begin{frame}[fragile]\frametitle{Specify figure/table/minipage size.}

\begin{itemize}
\item
  Prefer relative length to absolute length.
\item
  Prefer \lstinline{.45\textwidth} to \lstinline{2in}, \lstinline{3cm}.
\item
  \lstinline{\textheight}, \lstinline{\paperheight}, \lstinline{\paperwidth}
\end{itemize}

\end{frame}

\begin{frame}[fragile]\frametitle{Deal with subfigures.}

\begin{itemize}
\item
  Prefer \pkg{subcaption} package.
\item
  Use \pkg{subfig} with \pkg{IEEEtran} articles.
\item
  Do not use deprecated \pkg{subfigure} package.
\end{itemize}

\end{frame}


\begin{frame}[fragile]\frametitle{Compose math formulas.}

\begin{itemize}
\item
  Variables: use common notations, introduce them
\item
  Vectors/Matrices in bold: \pkg{bm}, \lstinline{\mathbf}
\item
  Non-variables should \alert{not} be \textit{italic}:
  \begin{itemize}
    \item |R_{\max}| $R_{\max}$
    \item |\text{SINR}| $\text{SINR}$
    \item |\mathrm{e}^{\mathrm{i}\pi}+1=0|
      $\mathrm{e}^{\mathrm{i}\pi}+1=0$
  \end{itemize}
\item
  Proper bracket size: |\left( \sum ... \right)|
\end{itemize}

\end{frame}

\begin{frame}[fragile]\frametitle{Handle long math formulas.}

\begin{itemize}
\item
  Define intermediate variables/functions to rephrase.
\item
  Divide it into multiple lines.
\item
  Environments: \env{multline}, \env{split}, \env{array}
\item
  Match bracket size: |\vphantom|
\item
  Details: \pkg{mathmode} documentation
\end{itemize}

\end{frame}

\begin{frame}[fragile]\frametitle{Write numbers with units.}

\begin{itemize}
\item
  \pkg{siunitx} package:
  \begin{itemize}
    \item |\SI{9.8}{m/s^2}| \SI{9.8}{m/s^2}
    \item |\SI{6.02e-23}{mol^{-1}}| \SI{6.02e-23}{mol^{-1}}
  \end{itemize}
\item
  If \pkg{siunitx} is not possible:
  \begin{itemize}
    \item Write \lstinline{5~km}, not \lstinline{5km}.\nocite{commonerrors}
  \end{itemize}
\end{itemize}

\end{frame}

\begin{frame}[fragile]
  \frametitle{Do not blindly copy from websites! I}

  \begin{itemize}
    \item Common errors~\cite{biberrors}:
      \begin{itemize}
        \item<1->
          Author names: \lstinline[style=wrong]|Owens, J.D.|
          |John D. Owens|
        \item<2->
          Title case of title, booktitle
\begin{lstlisting}[style=wrong]
booktitle = {Proceedings of the second workshop on Software radio implementation forum},
\end{lstlisting}
\begin{lstlisting}
booktitle = {Proceedings of the Second Workshop on Software Radio Implementation Forum},
\end{lstlisting}%
        \item<3->
          Booktitle word order
\begin{lstlisting}[style=wrong]
booktitle={Intelligent Vehicles Symposium (IV), 2011 IEEE},
\end{lstlisting}
\begin{lstlisting}
booktitle = {Proceedings of the 2011 IEEE Intelligent Vehicles Symposium},
\end{lstlisting}
\end{itemize}
\end{itemize}
\end{frame}
\begin{frame}[fragile]
  \frametitle{Do not blindly copy from websites! II}
  \begin{itemize}
    \item Common errors continued:
      \begin{itemize}
        \item
          Month, pages
\begin{lstlisting}[style=wrong]
month={june},
pages={195 -200},
\end{lstlisting}
\begin{lstlisting}
month=jun,
pages={195--200},
\end{lstlisting}
\pause
        \item
          DOI, URL
\begin{lstlisting}[style=wrong]
url = {http://doi.acm.org/10.1145/1964179.1964185},
doi = {http://doi.acm.org/10.1145/1964179.1964185},
\end{lstlisting}
\begin{lstlisting}
doi = {10.1145/1964179.1964185},
\end{lstlisting}
      \end{itemize}
    \pause
    \item
      Auto fix: \href{https://github.com/alick9188/fixbib}{FixBib
      for Greasemonkey}
  \end{itemize}

\end{frame}

\begin{frame}[fragile]\frametitle{Citation}

\begin{itemize}
\item
  Use tie (tilde, nbsp): \lstinline|Text goes~\cite{texbook}.|
\item
  Cite multiple references: |\cite{ref1,ref2,ref3}|
\pause
\item
  Citations in brackets are \alert{not} text words.
\end{itemize}

\end{frame}

\begin{frame}[fragile]\frametitle{How many times to run?}

  \begin{figure}[htbp]
    \centering
    \includegraphics[width=\textwidth]{bibcompilation}
    \caption{\BibTeX\ compilation process.}
  \end{figure}

\begin{itemize}
    \pause
\item Automatic compilation:
  \begin{itemize}
    \item \lstinline[style=cli]{latexmk}, \lstinline[style=cli]|texify|
    \item Makefile
  \end{itemize}
\end{itemize}

\end{frame}

\begin{frame}[fragile]\frametitle{Fonts}

\begin{itemize}
\item
  Font family~\cite{beamer}: serif or sans-serif?
\begin{itemize}
  \item
    Sans is preferred today.
  \item
    Serif may be used in conservative scenario.
  \item
    Math texts might be better in serif.
    |\usefonttheme[onlymath]{serif}|
\end{itemize}
\pause
\item
  Prefer colored to italic texts.
\begin{itemize}
  \item |\alert{dangerous}| \alert{dangerous}
  \item |\emph{dangerous}| \emph{dangerous}
\end{itemize}
\end{itemize}

\end{frame}

\begin{frame}[fragile]\frametitle{Fragile frames}
  \begin{itemize}
    \item |\verb+Lorem Lipsum+|, \env{verbatim}
    \item |\begin{frame}[fragile] ... \end{frame}|
  \end{itemize}
\end{frame}

\section{Summary}

\begin{frame}\frametitle{Summary}

\begin{itemize}
\item
  Details matter for aesthetic and high quality documents.
\pause
\item
  \LaTeX\ is evolving. The old will be replaced by the new.
\pause
\item
  Some work is tedious. Try automate it!
\pause
\item
  Happy \TeX{}ing!
\end{itemize}
\nocite{l2tabu,elements}

\end{frame}

\appendix
\section<presentation>*{\appendixname}

\begin{frame}[allowframebreaks]
  \frametitle<presentation>{More readings.}

  \printbibliography

\end{frame}

\begin{frame}
  \frametitle{Ask the Web}
  \begin{itemize}
    \item Where to ask?
      \begin{itemize}
        \item tex.sx
      \end{itemize}
    \item How to ask?
      \begin{itemize}
        \item MWE
      \end{itemize}
  \end{itemize}
\end{frame}

\begin{frame}
  \frametitle{Join}
  \begin{itemize}
    \item ThuThesis: \url{https://github.com/xueruini/thuthesis}
    \item ThuSlides
    \item TUNA: \url{http://tuna.tsinghua.edu.cn/}
  \end{itemize}
\end{frame}

\end{document}
%%% vim: set sw=2 isk+=\: et tw=70 formatoptions+=mB:
